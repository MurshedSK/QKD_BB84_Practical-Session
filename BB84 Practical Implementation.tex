\documentclass[11pt]{article}

    \usepackage[breakable]{tcolorbox}
    \usepackage{parskip} % Stop auto-indenting (to mimic markdown behaviour)
    

    % Basic figure setup, for now with no caption control since it's done
    % automatically by Pandoc (which extracts ![](path) syntax from Markdown).
    \usepackage{graphicx}
    % Keep aspect ratio if custom image width or height is specified
    \setkeys{Gin}{keepaspectratio}
    % Maintain compatibility with old templates. Remove in nbconvert 6.0
    \let\Oldincludegraphics\includegraphics
    % Ensure that by default, figures have no caption (until we provide a
    % proper Figure object with a Caption API and a way to capture that
    % in the conversion process - todo).
    \usepackage{caption}
    \DeclareCaptionFormat{nocaption}{}
    \captionsetup{format=nocaption,aboveskip=0pt,belowskip=0pt}

    \usepackage{float}
    \floatplacement{figure}{H} % forces figures to be placed at the correct location
    \usepackage{xcolor} % Allow colors to be defined
    \usepackage{enumerate} % Needed for markdown enumerations to work
    \usepackage{geometry} % Used to adjust the document margins
    \usepackage{amsmath} % Equations
    \usepackage{amssymb} % Equations
    \usepackage{textcomp} % defines textquotesingle
    % Hack from http://tex.stackexchange.com/a/47451/13684:
    \AtBeginDocument{%
        \def\PYZsq{\textquotesingle}% Upright quotes in Pygmentized code
    }
    \usepackage{upquote} % Upright quotes for verbatim code
    \usepackage{eurosym} % defines \euro

    \usepackage{iftex}
    \ifPDFTeX
        \usepackage[T1]{fontenc}
        \IfFileExists{alphabeta.sty}{
              \usepackage{alphabeta}
          }{
              \usepackage[mathletters]{ucs}
              \usepackage[utf8x]{inputenc}
          }
    \else
        \usepackage{fontspec}
        \usepackage{unicode-math}
    \fi

    \usepackage{fancyvrb} % verbatim replacement that allows latex
    \usepackage{grffile} % extends the file name processing of package graphics
                         % to support a larger range
    \makeatletter % fix for old versions of grffile with XeLaTeX
    \@ifpackagelater{grffile}{2019/11/01}
    {
      % Do nothing on new versions
    }
    {
      \def\Gread@@xetex#1{%
        \IfFileExists{"\Gin@base".bb}%
        {\Gread@eps{\Gin@base.bb}}%
        {\Gread@@xetex@aux#1}%
      }
    }
    \makeatother
    \usepackage[Export]{adjustbox} % Used to constrain images to a maximum size
    \adjustboxset{max size={0.9\linewidth}{0.9\paperheight}}

    % The hyperref package gives us a pdf with properly built
    % internal navigation ('pdf bookmarks' for the table of contents,
    % internal cross-reference links, web links for URLs, etc.)
    \usepackage{hyperref}
    % The default LaTeX title has an obnoxious amount of whitespace. By default,
    % titling removes some of it. It also provides customization options.
    \usepackage{titling}
    \usepackage{longtable} % longtable support required by pandoc >1.10
    \usepackage{booktabs}  % table support for pandoc > 1.12.2
    \usepackage{array}     % table support for pandoc >= 2.11.3
    \usepackage{calc}      % table minipage width calculation for pandoc >= 2.11.1
    \usepackage[inline]{enumitem} % IRkernel/repr support (it uses the enumerate* environment)
    \usepackage[normalem]{ulem} % ulem is needed to support strikethroughs (\sout)
                                % normalem makes italics be italics, not underlines
    \usepackage{soul}      % strikethrough (\st) support for pandoc >= 3.0.0
    \usepackage{mathrsfs}
    

    
    % Colors for the hyperref package
    \definecolor{urlcolor}{rgb}{0,.145,.698}
    \definecolor{linkcolor}{rgb}{.71,0.21,0.01}
    \definecolor{citecolor}{rgb}{.12,.54,.11}

    % ANSI colors
    \definecolor{ansi-black}{HTML}{3E424D}
    \definecolor{ansi-black-intense}{HTML}{282C36}
    \definecolor{ansi-red}{HTML}{E75C58}
    \definecolor{ansi-red-intense}{HTML}{B22B31}
    \definecolor{ansi-green}{HTML}{00A250}
    \definecolor{ansi-green-intense}{HTML}{007427}
    \definecolor{ansi-yellow}{HTML}{DDB62B}
    \definecolor{ansi-yellow-intense}{HTML}{B27D12}
    \definecolor{ansi-blue}{HTML}{208FFB}
    \definecolor{ansi-blue-intense}{HTML}{0065CA}
    \definecolor{ansi-magenta}{HTML}{D160C4}
    \definecolor{ansi-magenta-intense}{HTML}{A03196}
    \definecolor{ansi-cyan}{HTML}{60C6C8}
    \definecolor{ansi-cyan-intense}{HTML}{258F8F}
    \definecolor{ansi-white}{HTML}{C5C1B4}
    \definecolor{ansi-white-intense}{HTML}{A1A6B2}
    \definecolor{ansi-default-inverse-fg}{HTML}{FFFFFF}
    \definecolor{ansi-default-inverse-bg}{HTML}{000000}

    % common color for the border for error outputs.
    \definecolor{outerrorbackground}{HTML}{FFDFDF}

    % commands and environments needed by pandoc snippets
    % extracted from the output of `pandoc -s`
    \providecommand{\tightlist}{%
      \setlength{\itemsep}{0pt}\setlength{\parskip}{0pt}}
    \DefineVerbatimEnvironment{Highlighting}{Verbatim}{commandchars=\\\{\}}
    % Add ',fontsize=\small' for more characters per line
    \newenvironment{Shaded}{}{}
    \newcommand{\KeywordTok}[1]{\textcolor[rgb]{0.00,0.44,0.13}{\textbf{{#1}}}}
    \newcommand{\DataTypeTok}[1]{\textcolor[rgb]{0.56,0.13,0.00}{{#1}}}
    \newcommand{\DecValTok}[1]{\textcolor[rgb]{0.25,0.63,0.44}{{#1}}}
    \newcommand{\BaseNTok}[1]{\textcolor[rgb]{0.25,0.63,0.44}{{#1}}}
    \newcommand{\FloatTok}[1]{\textcolor[rgb]{0.25,0.63,0.44}{{#1}}}
    \newcommand{\CharTok}[1]{\textcolor[rgb]{0.25,0.44,0.63}{{#1}}}
    \newcommand{\StringTok}[1]{\textcolor[rgb]{0.25,0.44,0.63}{{#1}}}
    \newcommand{\CommentTok}[1]{\textcolor[rgb]{0.38,0.63,0.69}{\textit{{#1}}}}
    \newcommand{\OtherTok}[1]{\textcolor[rgb]{0.00,0.44,0.13}{{#1}}}
    \newcommand{\AlertTok}[1]{\textcolor[rgb]{1.00,0.00,0.00}{\textbf{{#1}}}}
    \newcommand{\FunctionTok}[1]{\textcolor[rgb]{0.02,0.16,0.49}{{#1}}}
    \newcommand{\RegionMarkerTok}[1]{{#1}}
    \newcommand{\ErrorTok}[1]{\textcolor[rgb]{1.00,0.00,0.00}{\textbf{{#1}}}}
    \newcommand{\NormalTok}[1]{{#1}}

    % Additional commands for more recent versions of Pandoc
    \newcommand{\ConstantTok}[1]{\textcolor[rgb]{0.53,0.00,0.00}{{#1}}}
    \newcommand{\SpecialCharTok}[1]{\textcolor[rgb]{0.25,0.44,0.63}{{#1}}}
    \newcommand{\VerbatimStringTok}[1]{\textcolor[rgb]{0.25,0.44,0.63}{{#1}}}
    \newcommand{\SpecialStringTok}[1]{\textcolor[rgb]{0.73,0.40,0.53}{{#1}}}
    \newcommand{\ImportTok}[1]{{#1}}
    \newcommand{\DocumentationTok}[1]{\textcolor[rgb]{0.73,0.13,0.13}{\textit{{#1}}}}
    \newcommand{\AnnotationTok}[1]{\textcolor[rgb]{0.38,0.63,0.69}{\textbf{\textit{{#1}}}}}
    \newcommand{\CommentVarTok}[1]{\textcolor[rgb]{0.38,0.63,0.69}{\textbf{\textit{{#1}}}}}
    \newcommand{\VariableTok}[1]{\textcolor[rgb]{0.10,0.09,0.49}{{#1}}}
    \newcommand{\ControlFlowTok}[1]{\textcolor[rgb]{0.00,0.44,0.13}{\textbf{{#1}}}}
    \newcommand{\OperatorTok}[1]{\textcolor[rgb]{0.40,0.40,0.40}{{#1}}}
    \newcommand{\BuiltInTok}[1]{{#1}}
    \newcommand{\ExtensionTok}[1]{{#1}}
    \newcommand{\PreprocessorTok}[1]{\textcolor[rgb]{0.74,0.48,0.00}{{#1}}}
    \newcommand{\AttributeTok}[1]{\textcolor[rgb]{0.49,0.56,0.16}{{#1}}}
    \newcommand{\InformationTok}[1]{\textcolor[rgb]{0.38,0.63,0.69}{\textbf{\textit{{#1}}}}}
    \newcommand{\WarningTok}[1]{\textcolor[rgb]{0.38,0.63,0.69}{\textbf{\textit{{#1}}}}}
    \makeatletter
    \newsavebox\pandoc@box
    \newcommand*\pandocbounded[1]{%
      \sbox\pandoc@box{#1}%
      % scaling factors for width and height
      \Gscale@div\@tempa\textheight{\dimexpr\ht\pandoc@box+\dp\pandoc@box\relax}%
      \Gscale@div\@tempb\linewidth{\wd\pandoc@box}%
      % select the smaller of both
      \ifdim\@tempb\p@<\@tempa\p@
        \let\@tempa\@tempb
      \fi
      % scaling accordingly (\@tempa < 1)
      \ifdim\@tempa\p@<\p@
        \scalebox{\@tempa}{\usebox\pandoc@box}%
      % scaling not needed, use as it is
      \else
        \usebox{\pandoc@box}%
      \fi
    }
    \makeatother

    % Define a nice break command that doesn't care if a line doesn't already
    % exist.
    \def\br{\hspace*{\fill} \\* }
    % Math Jax compatibility definitions
    \def\gt{>}
    \def\lt{<}
    \let\Oldtex\TeX
    \let\Oldlatex\LaTeX
    \renewcommand{\TeX}{\textrm{\Oldtex}}
    \renewcommand{\LaTeX}{\textrm{\Oldlatex}}
    % Document parameters
    % Document title
    \title{BB84 Practical Implementation}
    
    
    
    
    
    
    
% Pygments definitions
\makeatletter
\def\PY@reset{\let\PY@it=\relax \let\PY@bf=\relax%
    \let\PY@ul=\relax \let\PY@tc=\relax%
    \let\PY@bc=\relax \let\PY@ff=\relax}
\def\PY@tok#1{\csname PY@tok@#1\endcsname}
\def\PY@toks#1+{\ifx\relax#1\empty\else%
    \PY@tok{#1}\expandafter\PY@toks\fi}
\def\PY@do#1{\PY@bc{\PY@tc{\PY@ul{%
    \PY@it{\PY@bf{\PY@ff{#1}}}}}}}
\def\PY#1#2{\PY@reset\PY@toks#1+\relax+\PY@do{#2}}

\@namedef{PY@tok@w}{\def\PY@tc##1{\textcolor[rgb]{0.73,0.73,0.73}{##1}}}
\@namedef{PY@tok@c}{\let\PY@it=\textit\def\PY@tc##1{\textcolor[rgb]{0.24,0.48,0.48}{##1}}}
\@namedef{PY@tok@cp}{\def\PY@tc##1{\textcolor[rgb]{0.61,0.40,0.00}{##1}}}
\@namedef{PY@tok@k}{\let\PY@bf=\textbf\def\PY@tc##1{\textcolor[rgb]{0.00,0.50,0.00}{##1}}}
\@namedef{PY@tok@kp}{\def\PY@tc##1{\textcolor[rgb]{0.00,0.50,0.00}{##1}}}
\@namedef{PY@tok@kt}{\def\PY@tc##1{\textcolor[rgb]{0.69,0.00,0.25}{##1}}}
\@namedef{PY@tok@o}{\def\PY@tc##1{\textcolor[rgb]{0.40,0.40,0.40}{##1}}}
\@namedef{PY@tok@ow}{\let\PY@bf=\textbf\def\PY@tc##1{\textcolor[rgb]{0.67,0.13,1.00}{##1}}}
\@namedef{PY@tok@nb}{\def\PY@tc##1{\textcolor[rgb]{0.00,0.50,0.00}{##1}}}
\@namedef{PY@tok@nf}{\def\PY@tc##1{\textcolor[rgb]{0.00,0.00,1.00}{##1}}}
\@namedef{PY@tok@nc}{\let\PY@bf=\textbf\def\PY@tc##1{\textcolor[rgb]{0.00,0.00,1.00}{##1}}}
\@namedef{PY@tok@nn}{\let\PY@bf=\textbf\def\PY@tc##1{\textcolor[rgb]{0.00,0.00,1.00}{##1}}}
\@namedef{PY@tok@ne}{\let\PY@bf=\textbf\def\PY@tc##1{\textcolor[rgb]{0.80,0.25,0.22}{##1}}}
\@namedef{PY@tok@nv}{\def\PY@tc##1{\textcolor[rgb]{0.10,0.09,0.49}{##1}}}
\@namedef{PY@tok@no}{\def\PY@tc##1{\textcolor[rgb]{0.53,0.00,0.00}{##1}}}
\@namedef{PY@tok@nl}{\def\PY@tc##1{\textcolor[rgb]{0.46,0.46,0.00}{##1}}}
\@namedef{PY@tok@ni}{\let\PY@bf=\textbf\def\PY@tc##1{\textcolor[rgb]{0.44,0.44,0.44}{##1}}}
\@namedef{PY@tok@na}{\def\PY@tc##1{\textcolor[rgb]{0.41,0.47,0.13}{##1}}}
\@namedef{PY@tok@nt}{\let\PY@bf=\textbf\def\PY@tc##1{\textcolor[rgb]{0.00,0.50,0.00}{##1}}}
\@namedef{PY@tok@nd}{\def\PY@tc##1{\textcolor[rgb]{0.67,0.13,1.00}{##1}}}
\@namedef{PY@tok@s}{\def\PY@tc##1{\textcolor[rgb]{0.73,0.13,0.13}{##1}}}
\@namedef{PY@tok@sd}{\let\PY@it=\textit\def\PY@tc##1{\textcolor[rgb]{0.73,0.13,0.13}{##1}}}
\@namedef{PY@tok@si}{\let\PY@bf=\textbf\def\PY@tc##1{\textcolor[rgb]{0.64,0.35,0.47}{##1}}}
\@namedef{PY@tok@se}{\let\PY@bf=\textbf\def\PY@tc##1{\textcolor[rgb]{0.67,0.36,0.12}{##1}}}
\@namedef{PY@tok@sr}{\def\PY@tc##1{\textcolor[rgb]{0.64,0.35,0.47}{##1}}}
\@namedef{PY@tok@ss}{\def\PY@tc##1{\textcolor[rgb]{0.10,0.09,0.49}{##1}}}
\@namedef{PY@tok@sx}{\def\PY@tc##1{\textcolor[rgb]{0.00,0.50,0.00}{##1}}}
\@namedef{PY@tok@m}{\def\PY@tc##1{\textcolor[rgb]{0.40,0.40,0.40}{##1}}}
\@namedef{PY@tok@gh}{\let\PY@bf=\textbf\def\PY@tc##1{\textcolor[rgb]{0.00,0.00,0.50}{##1}}}
\@namedef{PY@tok@gu}{\let\PY@bf=\textbf\def\PY@tc##1{\textcolor[rgb]{0.50,0.00,0.50}{##1}}}
\@namedef{PY@tok@gd}{\def\PY@tc##1{\textcolor[rgb]{0.63,0.00,0.00}{##1}}}
\@namedef{PY@tok@gi}{\def\PY@tc##1{\textcolor[rgb]{0.00,0.52,0.00}{##1}}}
\@namedef{PY@tok@gr}{\def\PY@tc##1{\textcolor[rgb]{0.89,0.00,0.00}{##1}}}
\@namedef{PY@tok@ge}{\let\PY@it=\textit}
\@namedef{PY@tok@gs}{\let\PY@bf=\textbf}
\@namedef{PY@tok@ges}{\let\PY@bf=\textbf\let\PY@it=\textit}
\@namedef{PY@tok@gp}{\let\PY@bf=\textbf\def\PY@tc##1{\textcolor[rgb]{0.00,0.00,0.50}{##1}}}
\@namedef{PY@tok@go}{\def\PY@tc##1{\textcolor[rgb]{0.44,0.44,0.44}{##1}}}
\@namedef{PY@tok@gt}{\def\PY@tc##1{\textcolor[rgb]{0.00,0.27,0.87}{##1}}}
\@namedef{PY@tok@err}{\def\PY@bc##1{{\setlength{\fboxsep}{\string -\fboxrule}\fcolorbox[rgb]{1.00,0.00,0.00}{1,1,1}{\strut ##1}}}}
\@namedef{PY@tok@kc}{\let\PY@bf=\textbf\def\PY@tc##1{\textcolor[rgb]{0.00,0.50,0.00}{##1}}}
\@namedef{PY@tok@kd}{\let\PY@bf=\textbf\def\PY@tc##1{\textcolor[rgb]{0.00,0.50,0.00}{##1}}}
\@namedef{PY@tok@kn}{\let\PY@bf=\textbf\def\PY@tc##1{\textcolor[rgb]{0.00,0.50,0.00}{##1}}}
\@namedef{PY@tok@kr}{\let\PY@bf=\textbf\def\PY@tc##1{\textcolor[rgb]{0.00,0.50,0.00}{##1}}}
\@namedef{PY@tok@bp}{\def\PY@tc##1{\textcolor[rgb]{0.00,0.50,0.00}{##1}}}
\@namedef{PY@tok@fm}{\def\PY@tc##1{\textcolor[rgb]{0.00,0.00,1.00}{##1}}}
\@namedef{PY@tok@vc}{\def\PY@tc##1{\textcolor[rgb]{0.10,0.09,0.49}{##1}}}
\@namedef{PY@tok@vg}{\def\PY@tc##1{\textcolor[rgb]{0.10,0.09,0.49}{##1}}}
\@namedef{PY@tok@vi}{\def\PY@tc##1{\textcolor[rgb]{0.10,0.09,0.49}{##1}}}
\@namedef{PY@tok@vm}{\def\PY@tc##1{\textcolor[rgb]{0.10,0.09,0.49}{##1}}}
\@namedef{PY@tok@sa}{\def\PY@tc##1{\textcolor[rgb]{0.73,0.13,0.13}{##1}}}
\@namedef{PY@tok@sb}{\def\PY@tc##1{\textcolor[rgb]{0.73,0.13,0.13}{##1}}}
\@namedef{PY@tok@sc}{\def\PY@tc##1{\textcolor[rgb]{0.73,0.13,0.13}{##1}}}
\@namedef{PY@tok@dl}{\def\PY@tc##1{\textcolor[rgb]{0.73,0.13,0.13}{##1}}}
\@namedef{PY@tok@s2}{\def\PY@tc##1{\textcolor[rgb]{0.73,0.13,0.13}{##1}}}
\@namedef{PY@tok@sh}{\def\PY@tc##1{\textcolor[rgb]{0.73,0.13,0.13}{##1}}}
\@namedef{PY@tok@s1}{\def\PY@tc##1{\textcolor[rgb]{0.73,0.13,0.13}{##1}}}
\@namedef{PY@tok@mb}{\def\PY@tc##1{\textcolor[rgb]{0.40,0.40,0.40}{##1}}}
\@namedef{PY@tok@mf}{\def\PY@tc##1{\textcolor[rgb]{0.40,0.40,0.40}{##1}}}
\@namedef{PY@tok@mh}{\def\PY@tc##1{\textcolor[rgb]{0.40,0.40,0.40}{##1}}}
\@namedef{PY@tok@mi}{\def\PY@tc##1{\textcolor[rgb]{0.40,0.40,0.40}{##1}}}
\@namedef{PY@tok@il}{\def\PY@tc##1{\textcolor[rgb]{0.40,0.40,0.40}{##1}}}
\@namedef{PY@tok@mo}{\def\PY@tc##1{\textcolor[rgb]{0.40,0.40,0.40}{##1}}}
\@namedef{PY@tok@ch}{\let\PY@it=\textit\def\PY@tc##1{\textcolor[rgb]{0.24,0.48,0.48}{##1}}}
\@namedef{PY@tok@cm}{\let\PY@it=\textit\def\PY@tc##1{\textcolor[rgb]{0.24,0.48,0.48}{##1}}}
\@namedef{PY@tok@cpf}{\let\PY@it=\textit\def\PY@tc##1{\textcolor[rgb]{0.24,0.48,0.48}{##1}}}
\@namedef{PY@tok@c1}{\let\PY@it=\textit\def\PY@tc##1{\textcolor[rgb]{0.24,0.48,0.48}{##1}}}
\@namedef{PY@tok@cs}{\let\PY@it=\textit\def\PY@tc##1{\textcolor[rgb]{0.24,0.48,0.48}{##1}}}

\def\PYZbs{\char`\\}
\def\PYZus{\char`\_}
\def\PYZob{\char`\{}
\def\PYZcb{\char`\}}
\def\PYZca{\char`\^}
\def\PYZam{\char`\&}
\def\PYZlt{\char`\<}
\def\PYZgt{\char`\>}
\def\PYZsh{\char`\#}
\def\PYZpc{\char`\%}
\def\PYZdl{\char`\$}
\def\PYZhy{\char`\-}
\def\PYZsq{\char`\'}
\def\PYZdq{\char`\"}
\def\PYZti{\char`\~}
% for compatibility with earlier versions
\def\PYZat{@}
\def\PYZlb{[}
\def\PYZrb{]}
\makeatother


    % For linebreaks inside Verbatim environment from package fancyvrb.
    \makeatletter
        \newbox\Wrappedcontinuationbox
        \newbox\Wrappedvisiblespacebox
        \newcommand*\Wrappedvisiblespace {\textcolor{red}{\textvisiblespace}}
        \newcommand*\Wrappedcontinuationsymbol {\textcolor{red}{\llap{\tiny$\m@th\hookrightarrow$}}}
        \newcommand*\Wrappedcontinuationindent {3ex }
        \newcommand*\Wrappedafterbreak {\kern\Wrappedcontinuationindent\copy\Wrappedcontinuationbox}
        % Take advantage of the already applied Pygments mark-up to insert
        % potential linebreaks for TeX processing.
        %        {, <, #, %, $, ' and ": go to next line.
        %        _, }, ^, &, >, - and ~: stay at end of broken line.
        % Use of \textquotesingle for straight quote.
        \newcommand*\Wrappedbreaksatspecials {%
            \def\PYGZus{\discretionary{\char`\_}{\Wrappedafterbreak}{\char`\_}}%
            \def\PYGZob{\discretionary{}{\Wrappedafterbreak\char`\{}{\char`\{}}%
            \def\PYGZcb{\discretionary{\char`\}}{\Wrappedafterbreak}{\char`\}}}%
            \def\PYGZca{\discretionary{\char`\^}{\Wrappedafterbreak}{\char`\^}}%
            \def\PYGZam{\discretionary{\char`\&}{\Wrappedafterbreak}{\char`\&}}%
            \def\PYGZlt{\discretionary{}{\Wrappedafterbreak\char`\<}{\char`\<}}%
            \def\PYGZgt{\discretionary{\char`\>}{\Wrappedafterbreak}{\char`\>}}%
            \def\PYGZsh{\discretionary{}{\Wrappedafterbreak\char`\#}{\char`\#}}%
            \def\PYGZpc{\discretionary{}{\Wrappedafterbreak\char`\%}{\char`\%}}%
            \def\PYGZdl{\discretionary{}{\Wrappedafterbreak\char`\$}{\char`\$}}%
            \def\PYGZhy{\discretionary{\char`\-}{\Wrappedafterbreak}{\char`\-}}%
            \def\PYGZsq{\discretionary{}{\Wrappedafterbreak\textquotesingle}{\textquotesingle}}%
            \def\PYGZdq{\discretionary{}{\Wrappedafterbreak\char`\"}{\char`\"}}%
            \def\PYGZti{\discretionary{\char`\~}{\Wrappedafterbreak}{\char`\~}}%
        }
        % Some characters . , ; ? ! / are not pygmentized.
        % This macro makes them "active" and they will insert potential linebreaks
        \newcommand*\Wrappedbreaksatpunct {%
            \lccode`\~`\.\lowercase{\def~}{\discretionary{\hbox{\char`\.}}{\Wrappedafterbreak}{\hbox{\char`\.}}}%
            \lccode`\~`\,\lowercase{\def~}{\discretionary{\hbox{\char`\,}}{\Wrappedafterbreak}{\hbox{\char`\,}}}%
            \lccode`\~`\;\lowercase{\def~}{\discretionary{\hbox{\char`\;}}{\Wrappedafterbreak}{\hbox{\char`\;}}}%
            \lccode`\~`\:\lowercase{\def~}{\discretionary{\hbox{\char`\:}}{\Wrappedafterbreak}{\hbox{\char`\:}}}%
            \lccode`\~`\?\lowercase{\def~}{\discretionary{\hbox{\char`\?}}{\Wrappedafterbreak}{\hbox{\char`\?}}}%
            \lccode`\~`\!\lowercase{\def~}{\discretionary{\hbox{\char`\!}}{\Wrappedafterbreak}{\hbox{\char`\!}}}%
            \lccode`\~`\/\lowercase{\def~}{\discretionary{\hbox{\char`\/}}{\Wrappedafterbreak}{\hbox{\char`\/}}}%
            \catcode`\.\active
            \catcode`\,\active
            \catcode`\;\active
            \catcode`\:\active
            \catcode`\?\active
            \catcode`\!\active
            \catcode`\/\active
            \lccode`\~`\~
        }
    \makeatother

    \let\OriginalVerbatim=\Verbatim
    \makeatletter
    \renewcommand{\Verbatim}[1][1]{%
        %\parskip\z@skip
        \sbox\Wrappedcontinuationbox {\Wrappedcontinuationsymbol}%
        \sbox\Wrappedvisiblespacebox {\FV@SetupFont\Wrappedvisiblespace}%
        \def\FancyVerbFormatLine ##1{\hsize\linewidth
            \vtop{\raggedright\hyphenpenalty\z@\exhyphenpenalty\z@
                \doublehyphendemerits\z@\finalhyphendemerits\z@
                \strut ##1\strut}%
        }%
        % If the linebreak is at a space, the latter will be displayed as visible
        % space at end of first line, and a continuation symbol starts next line.
        % Stretch/shrink are however usually zero for typewriter font.
        \def\FV@Space {%
            \nobreak\hskip\z@ plus\fontdimen3\font minus\fontdimen4\font
            \discretionary{\copy\Wrappedvisiblespacebox}{\Wrappedafterbreak}
            {\kern\fontdimen2\font}%
        }%

        % Allow breaks at special characters using \PYG... macros.
        \Wrappedbreaksatspecials
        % Breaks at punctuation characters . , ; ? ! and / need catcode=\active
        \OriginalVerbatim[#1,codes*=\Wrappedbreaksatpunct]%
    }
    \makeatother

    % Exact colors from NB
    \definecolor{incolor}{HTML}{303F9F}
    \definecolor{outcolor}{HTML}{D84315}
    \definecolor{cellborder}{HTML}{CFCFCF}
    \definecolor{cellbackground}{HTML}{F7F7F7}

    % prompt
    \makeatletter
    \newcommand{\boxspacing}{\kern\kvtcb@left@rule\kern\kvtcb@boxsep}
    \makeatother
    \newcommand{\prompt}[4]{
        {\ttfamily\llap{{\color{#2}[#3]:\hspace{3pt}#4}}\vspace{-\baselineskip}}
    }
    

    
    % Prevent overflowing lines due to hard-to-break entities
    \sloppy
    % Setup hyperref package
    \hypersetup{
      breaklinks=true,  % so long urls are correctly broken across lines
      colorlinks=true,
      urlcolor=urlcolor,
      linkcolor=linkcolor,
      citecolor=citecolor,
      }
    % Slightly bigger margins than the latex defaults
    
    \geometry{verbose,tmargin=1in,bmargin=1in,lmargin=1in,rmargin=1in}
    
    

\begin{document}
    
    \maketitle
    
    

    
    \section{BB84 Simulation with Eve (Intercept-Resend
Attack)}\label{bb84-simulation-with-eve-intercept-resend-attack}

This notebook simulates 20 rounds of the BB84 protocol with an
eavesdropper, Eve, performing an intercept-and-resend attack.

    \subsubsection{1. Imports and Setup}\label{imports-and-setup}

First, we import \texttt{cirq} for quantum simulation, \texttt{numpy}
for random choices, and \texttt{pandas} to display the results in a
clean table. We also define our two bases: `R' (Rectilinear/Z-basis) and
`D' (Diagonal/X-basis).

    \begin{tcolorbox}[breakable, size=fbox, boxrule=1pt, pad at break*=1mm,colback=cellbackground, colframe=cellborder]
\prompt{In}{incolor}{1}{\boxspacing}
\begin{Verbatim}[commandchars=\\\{\}]
\PY{c+c1}{\PYZsh{} Install necessary libraries}
\PY{o}{!}pip\PY{+w}{ }install\PY{+w}{ }\PYZhy{}q\PY{+w}{ }cirq\PY{+w}{ }pandas
\PY{k+kn}{import}\PY{+w}{ }\PY{n+nn}{cirq}
\PY{k+kn}{import}\PY{+w}{ }\PY{n+nn}{numpy}\PY{+w}{ }\PY{k}{as}\PY{+w}{ }\PY{n+nn}{np}
\PY{k+kn}{import}\PY{+w}{ }\PY{n+nn}{pandas}\PY{+w}{ }\PY{k}{as}\PY{+w}{ }\PY{n+nn}{pd}

\PY{c+c1}{\PYZsh{} Define the two bases: Z (Rectilinear, R) and X (Diagonal, D)}
\PY{n}{BASIS\PYZus{}Z} \PY{o}{=} \PY{l+s+s1}{\PYZsq{}}\PY{l+s+s1}{R}\PY{l+s+s1}{\PYZsq{}} \PY{c+c1}{\PYZsh{} Z\PYZhy{}basis, Rectilinear}
\PY{n}{BASIS\PYZus{}X} \PY{o}{=} \PY{l+s+s1}{\PYZsq{}}\PY{l+s+s1}{D}\PY{l+s+s1}{\PYZsq{}} \PY{c+c1}{\PYZsh{} X\PYZhy{}basis, Diagonal}
\end{Verbatim}
\end{tcolorbox}

    \subsubsection{2. Helper Functions}\label{helper-functions}

We need functions to simulate the actions of Alice, Eve, and Bob.

\begin{itemize}
\tightlist
\item
  \texttt{prepare\_qubit}: Simulates preparing a qubit in one of the
  four BB84 states (\(|0\rangle, |1\rangle, |+\rangle, |-\rangle\)).
  This is used by Alice to create the initial qubit and by Eve to create
  her forged qubit.
\item
  \texttt{measure\_qubit}: Simulates measuring a qubit in either the `R'
  (Z) or `D' (X) basis. This is used by Eve to intercept the qubit and
  by Bob to get his final result.
\end{itemize}

    \begin{tcolorbox}[breakable, size=fbox, boxrule=1pt, pad at break*=1mm,colback=cellbackground, colframe=cellborder]
\prompt{In}{incolor}{2}{\boxspacing}
\begin{Verbatim}[commandchars=\\\{\}]
\PY{k}{def}\PY{+w}{ }\PY{n+nf}{prepare\PYZus{}qubit}\PY{p}{(}\PY{n}{bit}\PY{p}{,} \PY{n}{basis}\PY{p}{)}\PY{p}{:}
\PY{+w}{    }\PY{l+s+sd}{\PYZdq{}\PYZdq{}\PYZdq{}Alice or Eve prepares a qubit based on their bit and basis choice.\PYZdq{}\PYZdq{}\PYZdq{}}
    \PY{n}{qubit} \PY{o}{=} \PY{n}{cirq}\PY{o}{.}\PY{n}{LineQubit}\PY{p}{(}\PY{l+m+mi}{0}\PY{p}{)}
    
    \PY{k}{if} \PY{n}{bit} \PY{o}{==} \PY{l+m+mi}{0}\PY{p}{:}
        \PY{c+c1}{\PYZsh{} Prepare |0\PYZgt{} (for Z) or |+\PYZgt{} (for X)}
        \PY{k}{if} \PY{n}{basis} \PY{o}{==} \PY{n}{BASIS\PYZus{}Z}\PY{p}{:}
            \PY{c+c1}{\PYZsh{} |0\PYZgt{} state, no operation needed}
            \PY{k}{return} \PY{n}{qubit}\PY{p}{,} \PY{p}{[}\PY{p}{]}
        \PY{k}{else}\PY{p}{:}
            \PY{c+c1}{\PYZsh{} |+\PYZgt{} state}
            \PY{k}{return} \PY{n}{qubit}\PY{p}{,} \PY{p}{[}\PY{n}{cirq}\PY{o}{.}\PY{n}{H}\PY{p}{(}\PY{n}{qubit}\PY{p}{)}\PY{p}{]}
    \PY{k}{else}\PY{p}{:}
        \PY{c+c1}{\PYZsh{} Prepare |1\PYZgt{} (for Z) or |\PYZhy{}\PYZgt{} (for X)}
        \PY{k}{if} \PY{n}{basis} \PY{o}{==} \PY{n}{BASIS\PYZus{}Z}\PY{p}{:}
            \PY{c+c1}{\PYZsh{} |1\PYZgt{} state}
            \PY{k}{return} \PY{n}{qubit}\PY{p}{,} \PY{p}{[}\PY{n}{cirq}\PY{o}{.}\PY{n}{X}\PY{p}{(}\PY{n}{qubit}\PY{p}{)}\PY{p}{]}
        \PY{k}{else}\PY{p}{:}
            \PY{c+c1}{\PYZsh{} |\PYZhy{}\PYZgt{} state}
            \PY{k}{return} \PY{n}{qubit}\PY{p}{,} \PY{p}{[}\PY{n}{cirq}\PY{o}{.}\PY{n}{X}\PY{p}{(}\PY{n}{qubit}\PY{p}{)}\PY{p}{,} \PY{n}{cirq}\PY{o}{.}\PY{n}{H}\PY{p}{(}\PY{n}{qubit}\PY{p}{)}\PY{p}{]}

\PY{k}{def}\PY{+w}{ }\PY{n+nf}{measure\PYZus{}qubit}\PY{p}{(}\PY{n}{qubit}\PY{p}{,} \PY{n}{circuit}\PY{p}{,} \PY{n}{basis}\PY{p}{)}\PY{p}{:}
\PY{+w}{    }\PY{l+s+sd}{\PYZdq{}\PYZdq{}\PYZdq{}Bob or Eve measures the qubit in their chosen basis.\PYZdq{}\PYZdq{}\PYZdq{}}
    
    \PY{k}{if} \PY{n}{basis} \PY{o}{==} \PY{n}{BASIS\PYZus{}Z}\PY{p}{:}
        \PY{c+c1}{\PYZsh{} Z\PYZhy{}basis measurement is standard}
        \PY{n}{circuit}\PY{o}{.}\PY{n}{append}\PY{p}{(}\PY{n}{cirq}\PY{o}{.}\PY{n}{measure}\PY{p}{(}\PY{n}{qubit}\PY{p}{,} \PY{n}{key}\PY{o}{=}\PY{l+s+s1}{\PYZsq{}}\PY{l+s+s1}{result}\PY{l+s+s1}{\PYZsq{}}\PY{p}{)}\PY{p}{)}
    \PY{k}{else}\PY{p}{:}
        \PY{c+c1}{\PYZsh{} X\PYZhy{}basis measurement (apply Hadamard first)}
        \PY{n}{circuit}\PY{o}{.}\PY{n}{append}\PY{p}{(}\PY{n}{cirq}\PY{o}{.}\PY{n}{H}\PY{p}{(}\PY{n}{qubit}\PY{p}{)}\PY{p}{)}
        \PY{n}{circuit}\PY{o}{.}\PY{n}{append}\PY{p}{(}\PY{n}{cirq}\PY{o}{.}\PY{n}{measure}\PY{p}{(}\PY{n}{qubit}\PY{p}{,} \PY{n}{key}\PY{o}{=}\PY{l+s+s1}{\PYZsq{}}\PY{l+s+s1}{result}\PY{l+s+s1}{\PYZsq{}}\PY{p}{)}\PY{p}{)}
    
    \PY{c+c1}{\PYZsh{} Simulate the circuit}
    \PY{n}{simulator} \PY{o}{=} \PY{n}{cirq}\PY{o}{.}\PY{n}{Simulator}\PY{p}{(}\PY{p}{)}
    \PY{n}{result} \PY{o}{=} \PY{n}{simulator}\PY{o}{.}\PY{n}{run}\PY{p}{(}\PY{n}{circuit}\PY{p}{,} \PY{n}{repetitions}\PY{o}{=}\PY{l+m+mi}{1}\PY{p}{)}
    \PY{n}{measurement} \PY{o}{=} \PY{n}{result}\PY{o}{.}\PY{n}{measurements}\PY{p}{[}\PY{l+s+s1}{\PYZsq{}}\PY{l+s+s1}{result}\PY{l+s+s1}{\PYZsq{}}\PY{p}{]}\PY{p}{[}\PY{l+m+mi}{0}\PY{p}{]}\PY{p}{[}\PY{l+m+mi}{0}\PY{p}{]}
    \PY{k}{return} \PY{n}{measurement}
\end{Verbatim}
\end{tcolorbox}

    \subsubsection{3. Single Round
Simulation}\label{single-round-simulation}

This function combines the preparation and measurement steps to simulate
a single, full round of the protocol, from Alice to Eve to Bob.

\begin{enumerate}
\def\labelenumi{\arabic{enumi}.}
\tightlist
\item
  \textbf{Alice} randomly chooses a bit and a basis, then prepares a
  qubit.
\item
  \textbf{Eve} randomly chooses a basis, measures Alice's qubit, and
  gets a bit.
\item
  \textbf{Eve} then prepares a \emph{new} qubit based on the bit and
  basis she just used.
\item
  \textbf{Bob} randomly chooses a basis and measures the new qubit from
  Eve.
\end{enumerate}

    \begin{tcolorbox}[breakable, size=fbox, boxrule=1pt, pad at break*=1mm,colback=cellbackground, colframe=cellborder]
\prompt{In}{incolor}{3}{\boxspacing}
\begin{Verbatim}[commandchars=\\\{\}]
\PY{k}{def}\PY{+w}{ }\PY{n+nf}{run\PYZus{}bb84\PYZus{}round}\PY{p}{(}\PY{p}{)}\PY{p}{:}
\PY{+w}{    }\PY{l+s+sd}{\PYZdq{}\PYZdq{}\PYZdq{}Simulates one full round of BB84 with Alice, Eve, and Bob.\PYZdq{}\PYZdq{}\PYZdq{}}
    
    \PY{c+c1}{\PYZsh{} 1. ALICE}
    \PY{n}{alice\PYZus{}basis} \PY{o}{=} \PY{n}{np}\PY{o}{.}\PY{n}{random}\PY{o}{.}\PY{n}{choice}\PY{p}{(}\PY{p}{[}\PY{n}{BASIS\PYZus{}Z}\PY{p}{,} \PY{n}{BASIS\PYZus{}X}\PY{p}{]}\PY{p}{)}
    \PY{n}{alice\PYZus{}bit} \PY{o}{=} \PY{n}{np}\PY{o}{.}\PY{n}{random}\PY{o}{.}\PY{n}{choice}\PY{p}{(}\PY{p}{[}\PY{l+m+mi}{0}\PY{p}{,} \PY{l+m+mi}{1}\PY{p}{]}\PY{p}{)}
    \PY{n}{qubit}\PY{p}{,} \PY{n}{prep\PYZus{}ops} \PY{o}{=} \PY{n}{prepare\PYZus{}qubit}\PY{p}{(}\PY{n}{alice\PYZus{}bit}\PY{p}{,} \PY{n}{alice\PYZus{}basis}\PY{p}{)}
    \PY{n}{alice\PYZus{}circuit} \PY{o}{=} \PY{n}{cirq}\PY{o}{.}\PY{n}{Circuit}\PY{p}{(}\PY{n}{prep\PYZus{}ops}\PY{p}{)}
    
    \PY{c+c1}{\PYZsh{} 2. EVE (Intercept\PYZhy{}Resend Attack)}
    \PY{n}{eve\PYZus{}basis} \PY{o}{=} \PY{n}{np}\PY{o}{.}\PY{n}{random}\PY{o}{.}\PY{n}{choice}\PY{p}{(}\PY{p}{[}\PY{n}{BASIS\PYZus{}Z}\PY{p}{,} \PY{n}{BASIS\PYZus{}X}\PY{p}{]}\PY{p}{)}
    \PY{c+c1}{\PYZsh{} Eve measures Alice\PYZsq{}s qubit}
    \PY{n}{eve\PYZus{}circuit} \PY{o}{=} \PY{n}{alice\PYZus{}circuit}\PY{o}{.}\PY{n}{copy}\PY{p}{(}\PY{p}{)}
    \PY{n}{eve\PYZus{}bit} \PY{o}{=} \PY{n}{measure\PYZus{}qubit}\PY{p}{(}\PY{n}{qubit}\PY{p}{,} \PY{n}{eve\PYZus{}circuit}\PY{p}{,} \PY{n}{eve\PYZus{}basis}\PY{p}{)}
    
    \PY{c+c1}{\PYZsh{} Eve prepares a *new* qubit to send to Bob}
    \PY{n}{eve\PYZus{}qubit\PYZus{}to\PYZus{}bob}\PY{p}{,} \PY{n}{eve\PYZus{}prep\PYZus{}ops} \PY{o}{=} \PY{n}{prepare\PYZus{}qubit}\PY{p}{(}\PY{n}{eve\PYZus{}bit}\PY{p}{,} \PY{n}{eve\PYZus{}basis}\PY{p}{)}
    
    \PY{c+c1}{\PYZsh{} 3. BOB}
    \PY{n}{bob\PYZus{}basis} \PY{o}{=} \PY{n}{np}\PY{o}{.}\PY{n}{random}\PY{o}{.}\PY{n}{choice}\PY{p}{(}\PY{p}{[}\PY{n}{BASIS\PYZus{}Z}\PY{p}{,} \PY{n}{BASIS\PYZus{}X}\PY{p}{]}\PY{p}{)}
    \PY{n}{bob\PYZus{}circuit} \PY{o}{=} \PY{n}{cirq}\PY{o}{.}\PY{n}{Circuit}\PY{p}{(}\PY{n}{eve\PYZus{}prep\PYZus{}ops}\PY{p}{)}
    \PY{n}{bob\PYZus{}bit} \PY{o}{=} \PY{n}{measure\PYZus{}qubit}\PY{p}{(}\PY{n}{eve\PYZus{}qubit\PYZus{}to\PYZus{}bob}\PY{p}{,} \PY{n}{bob\PYZus{}circuit}\PY{p}{,} \PY{n}{bob\PYZus{}basis}\PY{p}{)}
    
    \PY{k}{return} \PY{p}{\PYZob{}}
        \PY{l+s+s2}{\PYZdq{}}\PY{l+s+s2}{Alice Basis}\PY{l+s+s2}{\PYZdq{}}\PY{p}{:} \PY{n}{alice\PYZus{}basis}\PY{p}{,}
        \PY{l+s+s2}{\PYZdq{}}\PY{l+s+s2}{Alice Bit}\PY{l+s+s2}{\PYZdq{}}\PY{p}{:} \PY{n}{alice\PYZus{}bit}\PY{p}{,}
        \PY{l+s+s2}{\PYZdq{}}\PY{l+s+s2}{Eve Basis}\PY{l+s+s2}{\PYZdq{}}\PY{p}{:} \PY{n}{eve\PYZus{}basis}\PY{p}{,}
        \PY{l+s+s2}{\PYZdq{}}\PY{l+s+s2}{Eve Bit}\PY{l+s+s2}{\PYZdq{}}\PY{p}{:} \PY{n}{eve\PYZus{}bit}\PY{p}{,}
        \PY{l+s+s2}{\PYZdq{}}\PY{l+s+s2}{Bob Basis}\PY{l+s+s2}{\PYZdq{}}\PY{p}{:} \PY{n}{bob\PYZus{}basis}\PY{p}{,}
        \PY{l+s+s2}{\PYZdq{}}\PY{l+s+s2}{Bob Bit}\PY{l+s+s2}{\PYZdq{}}\PY{p}{:} \PY{n}{bob\PYZus{}bit}
    \PY{p}{\PYZcb{}}
\end{Verbatim}
\end{tcolorbox}

    \subsubsection{4. Run the 20-Round
Simulation}\label{run-the-20-round-simulation}

Now we run the simulation 20 times and store the results in a
\texttt{pandas} DataFrame to view them clearly.

    \begin{tcolorbox}[breakable, size=fbox, boxrule=1pt, pad at break*=1mm,colback=cellbackground, colframe=cellborder]
\prompt{In}{incolor}{4}{\boxspacing}
\begin{Verbatim}[commandchars=\\\{\}]
\PY{n+nb}{print}\PY{p}{(}\PY{l+s+s2}{\PYZdq{}}\PY{l+s+s2}{\PYZhy{}\PYZhy{}\PYZhy{} Running 20\PYZhy{}Round BB84 Simulation (with Eve) \PYZhy{}\PYZhy{}\PYZhy{}}\PY{l+s+s2}{\PYZdq{}}\PY{p}{)}

\PY{n}{results} \PY{o}{=} \PY{p}{[}\PY{p}{]}
\PY{k}{for} \PY{n}{i} \PY{o+ow}{in} \PY{n+nb}{range}\PY{p}{(}\PY{l+m+mi}{20}\PY{p}{)}\PY{p}{:}
    \PY{n}{results}\PY{o}{.}\PY{n}{append}\PY{p}{(}\PY{n}{run\PYZus{}bb84\PYZus{}round}\PY{p}{(}\PY{p}{)}\PY{p}{)}

\PY{c+c1}{\PYZsh{} Display results in a clean table}
\PY{n}{df} \PY{o}{=} \PY{n}{pd}\PY{o}{.}\PY{n}{DataFrame}\PY{p}{(}\PY{n}{results}\PY{p}{)}
\PY{n}{df}\PY{o}{.}\PY{n}{index}\PY{o}{.}\PY{n}{name} \PY{o}{=} \PY{l+s+s2}{\PYZdq{}}\PY{l+s+s2}{Round}\PY{l+s+s2}{\PYZdq{}}
\PY{n+nb}{print}\PY{p}{(}\PY{n}{df}\PY{o}{.}\PY{n}{to\PYZus{}string}\PY{p}{(}\PY{p}{)}\PY{p}{)}
\end{Verbatim}
\end{tcolorbox}

    \begin{Verbatim}[commandchars=\\\{\}]
--- Running 20-Round BB84 Simulation (with Eve) ---
      Alice Basis  Alice Bit Eve Basis  Eve Bit Bob Basis  Bob Bit
Round
0               D          0         R        1         R        1
1               R          0         D        1         D        1
2               D          0         D        0         R        0
3               R          1         D        0         R        0
4               D          1         D        1         R        1
5               R          1         R        1         D        0
6               D          0         D        0         R        0
7               R          1         R        1         D        1
8               D          0         D        0         D        0
9               R          1         R        1         R        1
10              D          0         R        0         D        0
11              D          1         R        0         R        0
12              R          1         D        0         R        1
13              R          1         D        1         D        1
14              D          0         R        1         R        1
15              R          0         D        0         D        0
16              D          1         D        1         R        1
17              D          0         R        0         R        0
18              R          1         R        1         R        1
19              R          0         R        0         D        1
    \end{Verbatim}

    \subsubsection{5. Sifting and QBER
Calculation}\label{sifting-and-qber-calculation}

This is the final step, corresponding to \textbf{Questions 2.6 and 2.8}.

\begin{enumerate}
\def\labelenumi{\arabic{enumi}.}
\tightlist
\item
  \textbf{Sifting:} We simulate the public channel discussion by keeping
  \emph{only} the rounds where Alice and Bob's basis choices matched
  (`R'==`R' or `D'==`D').
\item
  \textbf{QBER Calculation:} We compare Alice's original bits and Bob's
  measured bits \emph{in the sifted rounds} to find the error rate.
\end{enumerate}

    \begin{tcolorbox}[breakable, size=fbox, boxrule=1pt, pad at break*=1mm,colback=cellbackground, colframe=cellborder]
\prompt{In}{incolor}{5}{\boxspacing}
\begin{Verbatim}[commandchars=\\\{\}]
\PY{n+nb}{print}\PY{p}{(}\PY{l+s+s2}{\PYZdq{}}\PY{l+s+se}{\PYZbs{}n}\PY{l+s+s2}{\PYZdq{}} \PY{o}{+} \PY{l+s+s2}{\PYZdq{}}\PY{l+s+s2}{=}\PY{l+s+s2}{\PYZdq{}}\PY{o}{*}\PY{l+m+mi}{70} \PY{o}{+} \PY{l+s+s2}{\PYZdq{}}\PY{l+s+se}{\PYZbs{}n}\PY{l+s+s2}{\PYZdq{}}\PY{p}{)}
\PY{n+nb}{print}\PY{p}{(}\PY{l+s+s2}{\PYZdq{}}\PY{l+s+s2}{\PYZhy{}\PYZhy{}\PYZhy{} Sifting and QBER Calculation \PYZhy{}\PYZhy{}\PYZhy{}}\PY{l+s+s2}{\PYZdq{}}\PY{p}{)}

\PY{c+c1}{\PYZsh{} Sifting: Keep only rounds where Alice and Bob\PYZsq{}s bases match}
\PY{n}{sifted\PYZus{}df} \PY{o}{=} \PY{n}{df}\PY{p}{[}\PY{n}{df}\PY{p}{[}\PY{l+s+s2}{\PYZdq{}}\PY{l+s+s2}{Alice Basis}\PY{l+s+s2}{\PYZdq{}}\PY{p}{]} \PY{o}{==} \PY{n}{df}\PY{p}{[}\PY{l+s+s2}{\PYZdq{}}\PY{l+s+s2}{Bob Basis}\PY{l+s+s2}{\PYZdq{}}\PY{p}{]}\PY{p}{]}\PY{o}{.}\PY{n}{copy}\PY{p}{(}\PY{p}{)}

\PY{k}{if} \PY{n+nb}{len}\PY{p}{(}\PY{n}{sifted\PYZus{}df}\PY{p}{)} \PY{o}{==} \PY{l+m+mi}{0}\PY{p}{:}
    \PY{n+nb}{print}\PY{p}{(}\PY{l+s+s2}{\PYZdq{}}\PY{l+s+s2}{No rounds had matching bases! (Unlikely, try running again)}\PY{l+s+s2}{\PYZdq{}}\PY{p}{)}
\PY{k}{else}\PY{p}{:}
    \PY{n+nb}{print}\PY{p}{(}\PY{l+s+sa}{f}\PY{l+s+s2}{\PYZdq{}}\PY{l+s+s2}{Bases matched for }\PY{l+s+si}{\PYZob{}}\PY{n+nb}{len}\PY{p}{(}\PY{n}{sifted\PYZus{}df}\PY{p}{)}\PY{l+s+si}{\PYZcb{}}\PY{l+s+s2}{ out of 20 rounds.}\PY{l+s+s2}{\PYZdq{}}\PY{p}{)}
    
    \PY{c+c1}{\PYZsh{} Compare Alice\PYZsq{}s and Bob\PYZsq{}s bits in the sifted rounds}
    \PY{n}{sifted\PYZus{}df}\PY{p}{[}\PY{l+s+s2}{\PYZdq{}}\PY{l+s+s2}{Error}\PY{l+s+s2}{\PYZdq{}}\PY{p}{]} \PY{o}{=} \PY{p}{(}\PY{n}{sifted\PYZus{}df}\PY{p}{[}\PY{l+s+s2}{\PYZdq{}}\PY{l+s+s2}{Alice Bit}\PY{l+s+s2}{\PYZdq{}}\PY{p}{]} \PY{o}{!=} \PY{n}{sifted\PYZus{}df}\PY{p}{[}\PY{l+s+s2}{\PYZdq{}}\PY{l+s+s2}{Bob Bit}\PY{l+s+s2}{\PYZdq{}}\PY{p}{]}\PY{p}{)}
    
    \PY{n}{alice\PYZus{}sifted\PYZus{}key} \PY{o}{=} \PY{l+s+s2}{\PYZdq{}}\PY{l+s+s2}{\PYZdq{}}\PY{o}{.}\PY{n}{join}\PY{p}{(}\PY{n}{sifted\PYZus{}df}\PY{p}{[}\PY{l+s+s2}{\PYZdq{}}\PY{l+s+s2}{Alice Bit}\PY{l+s+s2}{\PYZdq{}}\PY{p}{]}\PY{o}{.}\PY{n}{astype}\PY{p}{(}\PY{n+nb}{str}\PY{p}{)}\PY{p}{)}
    \PY{n}{bob\PYZus{}sifted\PYZus{}key} \PY{o}{=} \PY{l+s+s2}{\PYZdq{}}\PY{l+s+s2}{\PYZdq{}}\PY{o}{.}\PY{n}{join}\PY{p}{(}\PY{n}{sifted\PYZus{}df}\PY{p}{[}\PY{l+s+s2}{\PYZdq{}}\PY{l+s+s2}{Bob Bit}\PY{l+s+s2}{\PYZdq{}}\PY{p}{]}\PY{o}{.}\PY{n}{astype}\PY{p}{(}\PY{n+nb}{str}\PY{p}{)}\PY{p}{)}
    
    \PY{n+nb}{print}\PY{p}{(}\PY{l+s+sa}{f}\PY{l+s+s2}{\PYZdq{}}\PY{l+s+se}{\PYZbs{}n}\PY{l+s+s2}{Alice}\PY{l+s+s2}{\PYZsq{}}\PY{l+s+s2}{s Sifted Key: }\PY{l+s+si}{\PYZob{}}\PY{n}{alice\PYZus{}sifted\PYZus{}key}\PY{l+s+si}{\PYZcb{}}\PY{l+s+s2}{\PYZdq{}}\PY{p}{)}
    \PY{n+nb}{print}\PY{p}{(}\PY{l+s+sa}{f}\PY{l+s+s2}{\PYZdq{}}\PY{l+s+s2}{Bob}\PY{l+s+s2}{\PYZsq{}}\PY{l+s+s2}{s Sifted Key:   }\PY{l+s+si}{\PYZob{}}\PY{n}{bob\PYZus{}sifted\PYZus{}key}\PY{l+s+si}{\PYZcb{}}\PY{l+s+s2}{\PYZdq{}}\PY{p}{)}
    
    \PY{c+c1}{\PYZsh{} Calculate QBER}
    \PY{n}{num\PYZus{}errors} \PY{o}{=} \PY{n}{sifted\PYZus{}df}\PY{p}{[}\PY{l+s+s2}{\PYZdq{}}\PY{l+s+s2}{Error}\PY{l+s+s2}{\PYZdq{}}\PY{p}{]}\PY{o}{.}\PY{n}{sum}\PY{p}{(}\PY{p}{)}
    \PY{n}{num\PYZus{}sifted\PYZus{}bits} \PY{o}{=} \PY{n+nb}{len}\PY{p}{(}\PY{n}{sifted\PYZus{}df}\PY{p}{)}
    
    \PY{c+c1}{\PYZsh{} Avoid division by zero if no bits were sifted}
    \PY{k}{if} \PY{n}{num\PYZus{}sifted\PYZus{}bits} \PY{o}{\PYZgt{}} \PY{l+m+mi}{0}\PY{p}{:}
        \PY{n}{qber} \PY{o}{=} \PY{n}{num\PYZus{}errors} \PY{o}{/} \PY{n}{num\PYZus{}sifted\PYZus{}bits}
    \PY{k}{else}\PY{p}{:}
        \PY{n}{qber} \PY{o}{=} \PY{l+m+mi}{0} \PY{c+c1}{\PYZsh{} Or float(\PYZsq{}nan\PYZsq{})}
    
    \PY{n+nb}{print}\PY{p}{(}\PY{l+s+s2}{\PYZdq{}}\PY{l+s+se}{\PYZbs{}n}\PY{l+s+s2}{\PYZhy{}\PYZhy{}\PYZhy{} Final Result (for Q2.8) \PYZhy{}\PYZhy{}\PYZhy{}}\PY{l+s+s2}{\PYZdq{}}\PY{p}{)}
    \PY{n+nb}{print}\PY{p}{(}\PY{l+s+sa}{f}\PY{l+s+s2}{\PYZdq{}}\PY{l+s+s2}{Total Sifted Bits: }\PY{l+s+si}{\PYZob{}}\PY{n}{num\PYZus{}sifted\PYZus{}bits}\PY{l+s+si}{\PYZcb{}}\PY{l+s+s2}{\PYZdq{}}\PY{p}{)}
    \PY{n+nb}{print}\PY{p}{(}\PY{l+s+sa}{f}\PY{l+s+s2}{\PYZdq{}}\PY{l+s+s2}{Total Errors Found: }\PY{l+s+si}{\PYZob{}}\PY{n}{num\PYZus{}errors}\PY{l+s+si}{\PYZcb{}}\PY{l+s+s2}{\PYZdq{}}\PY{p}{)}
    
    \PY{k}{if} \PY{n}{num\PYZus{}sifted\PYZus{}bits} \PY{o}{\PYZgt{}} \PY{l+m+mi}{0}\PY{p}{:}
        \PY{n+nb}{print}\PY{p}{(}\PY{l+s+sa}{f}\PY{l+s+s2}{\PYZdq{}}\PY{l+s+s2}{QBER = }\PY{l+s+si}{\PYZob{}}\PY{n}{num\PYZus{}errors}\PY{l+s+si}{\PYZcb{}}\PY{l+s+s2}{ / }\PY{l+s+si}{\PYZob{}}\PY{n}{num\PYZus{}sifted\PYZus{}bits}\PY{l+s+si}{\PYZcb{}}\PY{l+s+s2}{ = }\PY{l+s+si}{\PYZob{}}\PY{n}{qber}\PY{l+s+si}{:}\PY{l+s+s2}{.2\PYZpc{}}\PY{l+s+si}{\PYZcb{}}\PY{l+s+s2}{\PYZdq{}}\PY{p}{)}
    \PY{k}{else}\PY{p}{:}
        \PY{n+nb}{print}\PY{p}{(}\PY{l+s+s2}{\PYZdq{}}\PY{l+s+s2}{QBER = N/A (no sifted bits)}\PY{l+s+s2}{\PYZdq{}}\PY{p}{)}
    
    \PY{n+nb}{print}\PY{p}{(}\PY{l+s+s2}{\PYZdq{}}\PY{l+s+se}{\PYZbs{}n}\PY{l+s+s2}{This QBER is the result of Eve}\PY{l+s+s2}{\PYZsq{}}\PY{l+s+s2}{s intercept\PYZhy{}resend attack.}\PY{l+s+s2}{\PYZdq{}}\PY{p}{)}
    \PY{n+nb}{print}\PY{p}{(}\PY{l+s+s2}{\PYZdq{}}\PY{l+s+s2}{The theoretical expected QBER is 25}\PY{l+s+s2}{\PYZpc{}}\PY{l+s+s2}{. Your simulation result should be close to this.}\PY{l+s+s2}{\PYZdq{}}\PY{p}{)}
\end{Verbatim}
\end{tcolorbox}

    \begin{Verbatim}[commandchars=\\\{\}]

======================================================================

--- Sifting and QBER Calculation ---
Bases matched for 6 out of 20 rounds.

Alice's Sifted Key: 101011
Bob's Sifted Key:   001011

--- Final Result (for Q2.8) ---
Total Sifted Bits: 6
Total Errors Found: 1
QBER = 1 / 6 = 16.67\%

This QBER is the result of Eve's intercept-resend attack.
The theoretical expected QBER is 25\%. Your simulation result should be close to
this.
    \end{Verbatim}

    \subsubsection{Conclusion}\label{conclusion}

This practical session successfully demonstrated the BB84 protocol.

\begin{itemize}
\tightlist
\item
  \textbf{Theory:} We mathematically confirmed that an intercept-resend
  attack introduces a \textbf{25\% QBER}, which results in a
  \textbf{negative key rate}, forcing the protocol to abort.
\item
  \textbf{Practice (Key Exchange):} We performed a 20-round key exchange
  (Q2.5) and successfully sifted a \textbf{7-bit key} (\texttt{0110010})
  with an ideal \textbf{QBER of 0\%} (Q2.6, Q2.8). This demonstrates the
  protocol's correctness in an error-free environment.
\item
  \textbf{Practice (Eavesdropping):} The experimental eavesdropping
  section (Q2.7) was replaced with a Cirq simulation. The simulation
  (results attached) produced a \textbf{16.67\% QBER}. This non-zero
  result confirms that Eve's presence introduces detectable errors,
  validating the security principle of the protocol. \#
\item
  \textbf{Final Insight:} The experiment highlights the fundamental
  difference between this classical-analog (which is insecure to
  beam-splitting) and a true quantum system, which relies on single
  photons and the no-cloning theorem to be secure.
\end{itemize}


    % Add a bibliography block to the postdoc
    
    
    
\end{document}
